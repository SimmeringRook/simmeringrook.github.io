\documentclass[]{report}
\usepackage{amssymb,amsmath}
\usepackage{hyperref}

\title{Exploring the Classification \\of Lagrangian Points as Frames of Reference\\  \large in Classical Mechanics and General Relativity}
\author{Thomas Knudson}
\date{24 Aug 2020}

\begin{document}
  \maketitle

  \begin{abstract}
    Abstract goes here.
    \end{abstract}

  \chapter{Problem Setup}
    \section{Problem Statement}
      Can we define a Lagrange Point as an inertial reference frame?

    \section{Goal}
      What are we trying to find? The order that we'll approach the problem in steps?

      \begin{enumerate}
       \item Step 1
       \item Step ???
       \begin{itemize}
         \item Sub Step
        \end{itemize}
       \item Profit!
        \end{enumerate}

    \section{Diagrams and Figures}
      Any diagrams that we created to respresent the situation or maybe plots? graphs? Visuals?

    \section{Applicable Concepts and Laws}
      \subsection{Newton's Laws of Motion}
        \begin{enumerate}
          \item In an inertial reference frame, an object either remains at rest or continues to move at a constant velocity, unless acted upon by a force.
          \item In an inertial frame of reference, the vector sum of the forces $F$ on an object is equal to the mass $m$ of that object multiplied by the acceleration $a$ of the object: $F = ma$. It is assumed that $m$ is constant.
          \item When one body exerts a force on a second body, the second body simultaneously exerts a force equal in magnitude and opposite in direction on the first body.
          \end{enumerate}
        Source: Wikipeda

      \subsection{Inertial Reference Frame}
        A reference frame in which Newton's First Law is obeyed. Another way to consider this is that there is a net force of 0 acting on the body in the frame. The size of a frame can be arbitrarily sized as long as the initial criteria is maintained. We can then say that objects traveling in these frames are said to have natural motion.

      \subsection{Lagrange Point}
        In celestial mechanics, the <strong>Lagranian points</strong> are orbital points near two large co-orbiting bodies. At the Lagrangian points, the gravitational forces of the two large bodies cancel out in such a way that a small object placed in orbit there is in equilibrium relative to the center of mass of the large bodies.

        Source: Wikipeda

      \subsection{General Relativity}
        There is no gravity, it is all curvature of spacetime.

  \chapter{Solution}

    \section{Using Classical Mechanics}

      \subsection{Trivial Proof}
        Due to the fact that we can arbitrarily size an inertial reference frame in terms of both space and time, we can consider a test particle that can be modeled as a point located at any of the Lagrange points. For an instant in time, that particle, by definition of its location meets the criteria for two co-orbiting massive objects.

      \subsection{Assumptions and Simplifications}
        Perhaps?

      \subsection{Proof}
        Convince yourself of the answer and publish your own work.

    \section{Using General Relativity}

      \subsection{Trivial Proof}
        Is there a trivial proof?

      \subsection{Assumptions and Simplifications}
        Maybe?

      \subsection{Proof}
        Convince yourself of the answer and publish your own work.

\end{document}
